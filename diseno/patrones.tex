\chapter{Patrones GoF}
\section{Patrones Creacionales}
Los patrones de diseño creacional resumen el proceso de creación de instancias. Ayudan a que un sistema sea independiente de cómo se crean, componen y representan sus objetos. Un patrón de creación de clases utiliza la herencia para variar la clase que está instanciada, mientras que un patrón de creación de objetos delegará la creación de instancias en otro objeto. Los patrones de creación se vuelven importantes a medida que los sistemas evolucionan para depender más de la composición de objetos que de la herencia de clase. Tal como sucede, el énfasis se aparta de la codificación rígida de un conjunto fijo de comportamientos para definir un conjunto más pequeño de comportamientos fundamentales que se pueden componer en cualquier cantidad de comportamientos más complejos. Por lo tanto, crear objetos con comportamientos particulares requiere algo más que simplemente crear una instancia de una clase. Hay dos temas recurrentes en estos patrones. Primero, todos encapsulan el conocimiento sobre las clases concretas que usa el sistema. En segundo lugar, ocultan cómo se crean y se crean las instancias de estas clases. Todo el sistema en general sabe acerca de los objetos es sus interfaces según lo definido por las clases abstractas. En consecuencia, los patrones creacionales le dan mucha flexibilidad en lo que se crea, quién lo crea, cómo se crea y cuándo. Le permiten configurar un sistema con objetos "producto" que varían ampliamente en estructura y funcionalidad. La configuración puede ser estática (es decir, especificada en tiempo de compilación) o dinámica (en tiempo de ejecución).\cite{gof}
\subsection{Patrón Prototype}
\subsubsection{Descripción}
Este patrón tiene or objetivo crear objetos prediseñados sin conocer detalles de cómo crearlos, en diversos casos el costo de crear un objeto elevado, especialmente cuando se deben especificar un gran conjunto de productos. Para este contexto, resulta conveniente, clonar dicho objeto, aprovecharse de la estructura original y ajustarle un nuevo propósito. 

\paragraph{Estructura}

\begin{figure}[th!]
	\centering
	\includegraphics[width=.7\linewidth]{imagenes/Patrones/Prototype.pdf}
	\caption{Estructura del patrón Prototype.\cite{gof}}	
\end{figure}

\paragraph{Actores}

\begin{itemize}
	\item \textbf{Prototype}: Declara la interface del objeto que se clona.
	\item \textbf{Concrete Prototype}: Implementa las operaciones para la clonación del objeto prototipo.
	\item \textbf{Cliente}: Accede a la generación de los objetos clonados.
\end{itemize}


\subsubsection{Caso de Uso}

\begin{figure}[th!]
	\centering
	\includegraphics[width=.7\linewidth]{imagenes/Patrones/Prototype_caso.pdf}
	\caption{Estructura del patrón Prototype caso de uso.\cite{gof}}	
\end{figure}



\subsection{Patrón Fabrica Abstracta}
\subsubsection{Descripción}
Este patrón al igual que todos los de tipo creacional, tiene por objetivo, solucionar le problema de código duro, el cúal es un escenario típico donde se crean instrucciones y objetos que al cerrarlos, no pueden cambiarse.
En dicho patrón, el cliente no quiere caer en código duro para la creación de productos concretos, por lo que lo hace a través de un conjunto de Fábrica Abstractas, que viene a ser la la envoltura, lo que facilita en gran medida, que cuando las familias de productos evoluciones, las familias se crean a través de fabricas abstractas.

\paragraph{Estructura}

\begin{figure}[th!]
	\centering
	\includegraphics[width=.7\linewidth]{imagenes/Patrones/Fabrica.pdf}
	\caption{Estructura del patrón Fabrica Abstracta.\cite{gof}}	
\end{figure}

\paragraph{Actores}

\begin{itemize}
	\item \textbf{Fabrica Abstracta}: Declara una interfaz para operaciones que crean objetos de productos abstractos.
	\item \textbf{Fabrica Concreta}: Implementa las operaciones para crear objetos productos concretos.
	\item \textbf{Producto Abstracto}: Declara la interfaz para un tipo de objeto producto.
	\item \textbf{Producto Concreto}: Define un objeto producto para que sea creado por la fabrica conrrespondiente. Implementa la interfaz Producto abstracto.
	\item \textbf{Cliente}: Solo usa la interfaces declaradas por las clases Fabrica Abstracta y Producto Abstracto.
	
\end{itemize}


\subsubsection{Caso de Uso}
\begin{figure}[th!]
	\centering
	\includegraphics[width=.7\linewidth]{imagenes/Patrones/Fabrica_caso.pdf}
	\caption{Estructura del patrón Fabrica Abstracta caso de uso.\cite{gof}}	
\end{figure}

\subsection{Patrón Fabrica}
\subsubsection{Descripción}
Define una interfaz para crear un objeto, pero deja que las subclases decidan cual clase instanciar. El Método Fábrica, permite a una clase delegar la instanciación a las subclases. Define también un constructor “virtual”.
El Método Fábrica realiza un diseño mas personalizable. Otros patrones de diseño requieren la creación de nuevas clases, en áreas donde el Método Fábrica solo requiere una nueva operación.


\paragraph{Estructura}

\begin{figure}[th!]
	\centering
	\includegraphics[width=.7\linewidth]{imagenes/Patrones/Builder.pdf}
	\caption{Estructura del patrón Fabrica.\cite{gof}}	
\end{figure}

\paragraph{Actores}

\begin{itemize}
	\item \textbf{Creador}: Esta clase(s) declaran el método fábrica, el cual retorna un objeto de tipo Producto. El Creador también puede definir una implementación por defecto del método fábrica que retorna un objeto por defecto del ProductoConcreto.
	\item \textbf{Creador Concreto}: Esta clase sobrescribe el método fábrica para retornar una instancia del ProductoConcreto.
	\item \textbf{Producto}: Esta clase define la interfaz de objetos que el método fábrica crea.
	\item \textbf{Producto Concreto}: Esta clase implementa la interfaz del Producto.	
\end{itemize}

\subsubsection{Caso de Uso}
\begin{figure}[th!]
	\centering
	\includegraphics[width=.7\linewidth]{imagenes/Patrones/Builder_caso.pdf}
	\caption{Estructura del patrón Fabrica caso de uso.\cite{gof}}	
\end{figure}



\section{Patrones de Comportamiento}
Los patrones de comportamiento se relacionan con los algoritmos y la asignación de responsabilidades entre los objetos. Los patrones de comportamiento describen no solo patrones de objetos o clases sino también los patrones de comunicación entre ellos. Estos patrones caracterizan el flujo de control complejo que es difícil de seguir en el tiempo de ejecución. Desvían su enfoque del flujo de control para permitirle concentrarse en el modo en que los objetos están interconectados. Los patrones de objetos de comportamiento utilizan la composición de objetos en lugar de la herencia. Algunos describen cómo un grupo de objetos similares cooperan para realizar una tarea que ningún objeto individual puede llevar a cabo por sí mismo. Un tema importante aquí es cómo los objetos de pares se conocen unos a otros. Los pares podrían mantener referencias explícitas entre sí, pero eso aumentaría su acoplamiento. En el extremo, cada objeto sabría sobre cada otro.\cite{gof}
\subsection{Patrón Strategy}

\subsubsection{Descripción}
Este patrón define una familia de algoritmos, encapsulándolos y haciéndolos intercambiables. Esto permite que el algoritmo varía independientemente del uso de los clientes.
\paragraph{Estructura}

\begin{figure}[th!]
	\centering
	\includegraphics[width=.7\linewidth]{imagenes/Patrones/Strategy.pdf}
	\caption{Estructura del patrón Strategy.\cite{gof}}	
\end{figure}

\paragraph{Actores}

\begin{itemize}
	\item \textbf{Contexto}: Esta clase es configurada con un objeto de EstrategiaConcreta, mantiene una referencia al objeto Estrategia y puede definir una interfaz que permite que Estrategia acceda a su información.
	\item \textbf{Estrategia}: Esta clase declara una interfaz común a todos los algoritmos soportados. El contexto utiliza está interfaz para llamar al algoritmo definido por la estrategia concreta.
	\item \textbf{Estrategia Concreta}: Esta clase implementa el algoritmo utilizando la interfaz de estrategia
\end{itemize}


\subsubsection{Caso de Uso}
\begin{figure}[th!]
	\centering
	\includegraphics[width=.7\linewidth]{imagenes/Patrones/Strategy_caso.pdf}
	\caption{Estructura del patrón Strategy caso de uso.\cite{gof}}	
\end{figure}

\subsection{Patrón State}

\subsubsection{Descripción}
Este patrón permite que un objeto modifique su comportamiento cada vez que cambie su estado interno. Sirve para modelar el estado de máquinas de un objeto y con ello su ciclo de vida.

\paragraph{Estructura}

\begin{figure}[th!]
	\centering
	\includegraphics[width=.7\linewidth]{imagenes/Patrones/Sate.pdf}
	\caption{Estructura del patrón State.\cite{gof}}	
\end{figure}

\paragraph{Actores}

\begin{itemize}
	\item \textbf{Contexto}: Define la interfaz de interés para los clientes. Mantiene una instancia de una subclase de EstadoConcreto que define el estado actual.
	\item \textbf{Estado}: Define una interfaz para encapsular el comportamiento asociado con un determinado estado del Contexto.
	\item \textbf{Estado Concreto}: Cada subclase implementa un comportamiento asociado con un estado del Contexto.
\end{itemize}


\subsubsection{Caso de Uso}
\begin{figure}[th!]
	\centering
	\includegraphics[width=.7\linewidth]{imagenes/Patrones/State_caso.pdf}
	\caption{Estructura del patrón State caso de uso.\cite{gof}}	
\end{figure}
\subsection{Patrón Cadena de Responsabilidad}

\subsubsection{Descripción}
Este patrón evita acoplar el emisor de una petición a su receptor, dando a más de un objeto la posibilidad de responder a la petición. Encadena los objetos receptores y pasa la petción a través de la cadena hasta que es procesada por algún objeto.
Este patrón tiene uso cuando más de un objeto puede resolver una petición, conformados además dinámicamente, definiendo el protocolo de respuesta.

\paragraph{Estructura}

\begin{figure}[th!]
	\centering
	\includegraphics[width=.7\linewidth]{imagenes/Patrones/Cadena.pdf}
	\caption{Estructura del patrón Cadena de Responsabilidad.\cite{gof}}	
\end{figure}

\paragraph{Actores}

\begin{itemize}
	\item \textbf{Manejador}: Define una interfaz para tratar las peticiones
	\item \textbf{Manejador Concreto}: Trata las peticiones de la que es responsable. Puede acceder a su sucesor. Si el manejadorConcreto puede manejar la petición, lo hace; en caso contrario la reenvía a su sucesor.
	\item \textbf{Cliente}: Inicializa la petición a un objeto ManejadorConcreto  de la cadena.
\end{itemize}


\subsubsection{Caso de Uso}
\begin{figure}[th!]
	\centering
	\includegraphics[width=.7\linewidth]{imagenes/Patrones/Cadena_caso.pdf}
	\caption{Estructura del patrón Cadena de Responsabilidad caso de uso.\cite{gof}}	
\end{figure}

\section{Patrones Estructurales}
Los patrones estructurales se refieren a cómo se componen las clases y los objetos para formar estructuras más grandes. Los patrones de clases estructurales utilizan la herencia para componer interfaces o implementaciones. Como ejemplo simple, considere cómo la herencia múltiple mezcla dos o más clases en una sola. El resultado es una clase que combina las propiedades de sus clases principales. Este patrón es particularmente útil para hacer que las bibliotecas de clases desarrolladas independientemente trabajen juntas. En lugar de componer interfaces o implementaciones, los patrones de objetos estructurales describen formas de componer objetos para realizar nuevas funcionalidades. La flexibilidad añadida de la composición de objetos proviene de la capacidad de cambiar la composición en tiempo de ejecución, lo que es imposible con la composición de clase estática.\cite{gof}
\subsection{Patrón Bridge}

\subsubsection{Descripción}

\paragraph{Estructura}

\begin{figure}[th!]
	\centering
	\includegraphics[width=.7\linewidth]{imagenes/Patrones/Bridge.pdf}
	\caption{Estructura del patrón Bridge.\cite{gof}}	
\end{figure}

\paragraph{Actores}

\begin{itemize}
	\item \textbf{Proxy}: Controla el acceso al sujeto real y puede ser responsable de crearlo y eliminarlo.
	\item \textbf{Sujeto}:Define la interfaz común para las clase SujetoReal y Proxy para que un Proxy se pueda usar en cualquier lugar donde se espere un SujetoReal.
	\item \textbf{Sujeto Real}:Define el objeto real al cual representa el Proxy.
\end{itemize}


\subsubsection{Caso de Uso}
\begin{figure}[th!]
	\centering
	\includegraphics[width=.7\linewidth]{imagenes/Patrones/Bridge_caso.pdf}
	\caption{Estructura del patrón Bridge caso de uso.\cite{gof}}	
\end{figure}
\subsection{Patrón Proxy}

\subsubsection{Descripción}
El patron Proxy es un patrón que permite controlar el acceso a un objeto, esto  mediante una entidad intermediara, por lo cual se puede diferir el costo total de la creación de un objeto hasta que realmente necesitemos usarlo, buscando finalmente optimizar tanto el uso de recursos computacionales como los tiempos de carga.

Tiene como aplicación:

\begin{itemize}
	\item Un proxy remoto puede ocultar el hecho de que un objeto reside en un espacio de direcciones diferente.
	\item Un proxy virtual puede realizar optimizaciones, como crear un objeto a pedido.
	\item Los proxies de protección y referencias inteligentes permiten darle diferentes permisos de objetoa a los que así lo necesitan.
\end{itemize}

\paragraph{Estructura}

\begin{figure}[th!]
	\centering
	\includegraphics[width=.7\linewidth]{imagenes/Patrones/estructura_Proxy.pdf}
	\caption{Estructura del patrón Proxy.\cite{gof}}	
\end{figure}

\paragraph{Actores}

\begin{itemize}
	\item \textbf{Proxy}: Controla el acceso al sujeto real y puede ser responsable de crearlo y eliminarlo.
	\item \textbf{Sujeto}:Define la interfaz común para las clase SujetoReal y Proxy para que un Proxy se pueda usar en cualquier lugar donde se espere un SujetoReal.
	\item \textbf{Sujeto Real}:Define el objeto real al cual representa el Proxy.
\end{itemize}


\subsubsection{Caso de Uso}


