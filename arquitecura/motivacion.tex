\chapter{Motivación}


\section{Introducción}
En las capas anteriores hemos visto como la estructura tecnológica le da soporte a la arquitectura empresarial, pero en esta capa vamos a visualizar y a modelar las razones o motivaciones que subyacen al diseño de esta, ya que son estas motivaciones las que restringen y van guiando el diseño de esta.
Las motivaciones reales están representadas por metas, principios, requisitos y restricciones.Las metas buscan representar algún resultado deseado - o fin - que una parte interesada desea lograr; por ejemplo, aumentar la satisfacción del cliente. 
Los principios y requisitos representan las propiedades deseadas de soluciones  o medios  mediante los cuales se pueden alcanzar los objetivos. 
Los principios son una serie de normativas  o pautas que guían el diseño de todas las soluciones posibles en un contexto dado. Y finalmente los requisitos representan declaraciones formales de necesidad, expresada por las partes interesadas, que debe cumplir la arquitectura o las soluciones siendo estas últimas inamovibles.


\section{Punto de Vista de StakeHolder}
\subsection{Descripción}
El punto de vista de infraestructura contiene los elementos de hardware y de software correspondientes a la infraestructura que soporta la capa de aplicación, en este diagrama podemos observar elementos ales como dispositivos físicos, redes o sistemas de software tales como: Bases de datos o sistemas operativos.

\subsubsection{Metamodelo}
\begin{figure}[H]
	\centering
	\includegraphics[width=1.0\textwidth]{imagenes/Metamodelos/Motivacion/meta_Stakeholder.pdf}
	\caption{Metamodelo: Punto de Vista de StakeHolder}
	\label{fig:gap_analysis}
\end{figure}

\subsubsection{Caso de Estudio}


\begin{figure}[H]
	\centering
	\includegraphics[width=1.0\textwidth]{imagenes/Caso_Estudio/Motivacion/Stakeholder.PDF}
	\caption{Caso de estudio: Punto de Vista de StakeHolder.}
	\label{fig:gap_analysis}
\end{figure}



\section{Punto de Vista de Realización de Objetivos}
\subsection{Descripción}

\subsubsection{Metamodelo}
\begin{figure}[H]
	\centering
	\includegraphics[width=1.0\textwidth]{imagenes/Metamodelos/Motivacion/meta_Realizacion_Objetivos.pdf}
	\caption{Metamodelo: Punto de Vista de Realización de Objetivos}
	\label{fig:gap_analysis}
\end{figure}

\subsubsection{Caso de Estudio}


\begin{figure}[H]
	\centering
	\includegraphics[width=1.0\textwidth]{imagenes/Caso_Estudio/Motivacion/Realizacion_Objetivos.PDF}
	\caption{Caso de estudio: Punto de Vista de Realización de Objetivos.}
	\label{fig:gap_analysis}
\end{figure}

\section{Punto de Vista de Contribución}
\subsection{Descripción}

\subsubsection{Metamodelo}
\begin{figure}[H]
	\centering
	\includegraphics[width=1.0\textwidth]{imagenes/Metamodelos/Motivacion/meta_Contribucon.pdf}
	\caption{Metamodelo: Punto de Vista de Contribución.}
	\label{fig:gap_analysis}
\end{figure}

\subsubsection{Caso de Estudio}


\begin{figure}[H]
	\centering
	\includegraphics[width=1.0\textwidth]{imagenes/Caso_Estudio/Motivacion/Contribucion.PDF}
	\caption{Caso de estudio: Punto de Vista de Contribución.}
	\label{fig:gap_analysis}
\end{figure}

\section{Punto de Vista de Principios}
\subsection{Descripción}

\subsubsection{Metamodelo}
\begin{figure}[H]
	\centering
	\includegraphics[width=1.0\textwidth]{imagenes/Metamodelos/Motivacion/meta_Principios.pdf}
	\caption{Metamodelo: Punto de Vista de Principios.}
	\label{fig:gap_analysis}
\end{figure}

\subsubsection{Caso de Estudio}


\begin{figure}[H]
	\centering
	\includegraphics[width=1.0\textwidth]{imagenes/Caso_Estudio/Motivacion/Principios.PDF}
	\caption{Caso de estudio: Punto de Vista de Principios.}
	\label{fig:gap_analysis}
\end{figure}

\section{Punto de Vista de Realización de Requerimientos}
\subsection{Descripción}

\subsubsection{Metamodelo}
\begin{figure}[H]
	\centering
	\includegraphics[width=1.0\textwidth]{imagenes/Metamodelos/Motivacion/meta_Realizacion_Requerimientos.pdf}
	\caption{Metamodelo: Punto de Vista de Realización de Requerimientos.}
	\label{fig:gap_analysis}
\end{figure}

\subsubsection{Caso de Estudio}


\begin{figure}[H]
	\centering
	\includegraphics[width=1.0\textwidth]{imagenes/Caso_Estudio/Motivacion/Realizacion_Requerimientos.PDF}
	\caption{Caso de estudio: Punto de Vista de Realización de Requerimientos.}
	\label{fig:gap_analysis}
\end{figure}


\section{Punto de Vista de Motivación}
\subsection{Descripción}

\subsubsection{Metamodelo}
\begin{figure}[H]
	\centering
	\includegraphics[width=1.0\textwidth]{imagenes/Metamodelos/Motivacion/meta_Motivacion.pdf}
	\caption{Metamodelo: Punto de Vista de Motivación.}
	\label{fig:gap_analysis}
\end{figure}

\subsubsection{Caso de Estudio}


\begin{figure}[H]
	\centering
	\includegraphics[width=1.0\textwidth]{imagenes/Caso_Estudio/Motivacion/Motivacion.PDF}
	\caption{Caso de estudio: Punto de Vista de Motivación.}
	\label{fig:gap_analysis}
\end{figure}


