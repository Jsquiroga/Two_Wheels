\chapter{Prefacio}

En la actualidad, el software es considerado como una herramienta inherente y en ciertos casos fundamental para el desarrollo de algunas actividades tanto de carácter cotidiano como en el ámbito empresarial que demanda una sociedad industrial y globalizada como la actual. En este sentido, la necesidad de construir productos software de alta calidad y de manera óptima es cada vez mayor, demandando así tanto a las empresas como a los profesionales afines adoptar estándares y prácticas que se ajusten a  y les permitan dar cumplimiento a los parámetros de calidad que demanda este sector.\\

El presente documento identifica, describe e implementa el método de desarrollo arquitectural \textbf{ADM}, esto desde una perspectiva empresarial y mediante el lenguaje de modelado \textbf{Archimate} lo cual nos  permite especificar y sustentar con un alto nivel de detalle como el proceso de software está altamente ligado a la organización y su estructura. Así mismo y adoptando un enfoque práctico a lo largo y ancho del documento se trabajan y usan todas estas herramientas y conceptos buscando generar una solución de software (producto final) que responde a la administración de un parqueadero de bicicletas para una empresa denominada \textit{\textbf{Two Wheels Parking}} .

