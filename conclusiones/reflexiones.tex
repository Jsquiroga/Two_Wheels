\chapter{Reflexiones}

\begin{itemize}
	\item Aplicar un método de diseño como el ADM o alguna metodología de desarrollo de software en principio puede parecer complicada, pero por el contrario son herramientas que simplifican, sistematizan y da un orden al proceso de creación de software.
	\item De manera general la arquitectura empresarial representa una herramienta de trabajo poderosa, ya que el uso de esta nos da un enfoque y una visión clara del cliente tanto a alto como a bajo nivel, lo que nos ayuda a identificar oportunidades y riesgos permitiéndonos abordar el problema de forma correcta y así optimizar el proceso de diseño de software.
	\item La implementación de los puntos de vista en cada una de las capas de arquitectura nos permitió sintetizar y representar de manera sencilla cada uno de los escenarios  y soluciones planteadas, convirtiéndose así en una alternativa clave de comunicación entre el cliente y el arquitecto.
	\item Durante el desarrollo del proyecto se evidencia que la codificación si bien es una de las fases de desarrollo de software de lejos es la más importante, esto contrapuesto a nuestra mirada inicial previa al desarrollo de este documento.
	\item El uso de los patrones de diseño dentro de un proyecto no es obligatoria, pero la implementación y buen conocimiento de ellos le brinda al arquitecto una forma ordenada de diseñar y responder a problemas específicos de manera rápida.
\end{itemize}